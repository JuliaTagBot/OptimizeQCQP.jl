\documentclass[letterpaper,abstracton]{scrartcl}
\usepackage{amssymb,amsmath}

\usepackage{graphicx}
%\usepackage{fancyhdr}
\usepackage{authblk}
\usepackage{hyperref}
\usepackage[comma,super,compress]{natbib}
\usepackage{xspace}

% Set up headers and footers
%\pagestyle{fancy}

\def\eq#1{Eq.~\eqref{eq:#1}}
\def\eqs#1{Eqs.~\eqref{eq:#1}}
\def\figref#1{Fig.~\ref{fig:#1}}
\newcommand{\ud}{\mathrm{d}}

% Put any new macros here
\newcommand{\bfx}{\mathbf{x}}
\newcommand{\bfg}{\mathbf{g}}
\newcommand{\bfQ}{\mathbf{Q}}

\begin{document}

\title{My Title}

%\author[1]{Someone else}
\author[1,*]{Timothy E. Holy}
\affil[1]{Department of Anatomy \& Neurobiology, Washington University School of Medicine, St. Louis, Missouri}
\affil[*]{To whom correspondence should be addressed. Email: \url{holy@wustl.edu}}

%\lhead{\emph{Running head}}
%\chead{}
%\rhead{\thepage}
%\lfoot{}
%\cfoot{}
%\rfoot{}

\date{\today}

\maketitle

\begin{abstract}
Here's the abstract
\end{abstract}

\section{Simple case}

Consider the problem
\begin{subequations}
  \begin{align}
    \textrm{minimize } &\phantom{=} \frac{1}{2} \bfx^T \bfQ_0 \bfx + \bfg_0^T \bfx \\
    \textrm{subject to } &\phantom{=} \frac{1}{2} \bfx^T \bfQ_1 \bfx \le c \label{eq:constraint1}
  \end{align}
\end{subequations}
for positive-semidefinite $\bfQ_1$.
Following chapter 4.3 of Nocedal \& Wright, consider $\bfQ_0 + \lambda \bfQ_1$ and assume that $\lambda$ is big enough to make this sum positive-definite.
We seek a $\lambda$ such that \eq{constraint1} is satisfied at
\begin{equation}
  \label{eq:xdef1}
  \bfx = (\bfQ_0 + \lambda \bfQ_1)^{-1} (-\bfg_0).
\end{equation}
Consider
\begin{equation}
  \phi_2(\lambda) = \frac{1}{c} - \frac{2}{\bfx^T \bfQ_1 \bfx},
\end{equation}
which exhibits a root at $\lambda$ such that \eq{constraint1} becomes an equality.
We can therefore update $\lambda$ by Newton's method
\begin{equation}
  \lambda^{(l+1)} = \lambda^{(l)} - \frac{\phi_2(\lambda)}{\phi_2'(\lambda)}.
\end{equation}
Since
\begin{equation}
  \phi_2'(\lambda) = \frac{4 \bfx^T \bfQ_1 \frac{\ud \bfx}{\ud \lambda}}{\left(\bfx^T\bfQ_1\bfx\right)^2},
\end{equation}
we have
\begin{equation}
  \frac{\phi_2(\lambda)}{\phi_2'(\lambda)} = \frac{\bfx^T \bfQ_1 \bfx}{2\bfx^T\bfQ_1\frac{\ud \bfx}{\ud \lambda}} \frac{\frac{1}{2}\bfx^T\bfQ_1\bfx - c}{c}.
\end{equation}
From \eq{xdef1}, we have
\begin{align}
  \frac{\ud \bfx}{\ud \lambda} &= - (\bfQ_0+\lambda\bfQ_1)^{-1}\bfQ_1(\bfQ_0+\lambda\bfQ_1)^{-1}(-\bfg_0) \\
  &= - (\bfQ_0+\lambda\bfQ_1)^{-1}\bfQ_1 \bfx.
\end{align}

\section{Cholesky factorization of symmetric tridiagonal matrices}

In this case the factorization is simple:
\begin{subequations}
  \label{eq:cholA}
  \begin{align}
    A_{ii} &= L_{i,i-1}^2 + L_{ii}^2 \\
    A_{i+1,i} &= L_{i+1,i}L_{ii}.
  \end{align}
\end{subequations}
We initiate it with $L_{11} = \sqrt{A_{11}}$, and then iterate
\begin{subequations}
  \label{eq:cholL}
  \begin{align}
    L_{i+1,i} &= \frac{A_{i+1,i}}{L_{ii}}; \\
    L_{i+1,i+1} &= \sqrt{A_{i+1,i+1} - L_{i+1,i}^2}.
  \end{align}
\end{subequations}
% Note that \eqs{cholA} imply that $A_{ii} - 2A_{i,i-1} = (L_{i,i-1} - L_{ii})^2 \ge 0$,
% consequently we can inspect the matrix and easily find a lower limit for what we need to add to the diagonal to avoid violating this requirement.
\url{http://www.mat.uc.pt/~cmf/papers/tri_positive_definite_revisited.pdf} contains a number of potential conditions that could be leveraged for guessing an initial $\lambda$, for example
\begin{equation}
  4 \cos^2\left(\frac{\pi}{n+1}\right) b_i^2 < a_i a_{i+1},
\end{equation}
where $b_i$ are the off-diagonal values and $a_i$ the diagonals. This results in a quadratic inequality for $\lambda$.

\begin{figure}[h]
\caption{
What a figure!
\label{fig:example}
}
\end{figure}

\begin{table}[h]
\caption{
Who needs tables?
\label{tab:dumbtable}
}
\end{table}


\clearpage

\end{document}
